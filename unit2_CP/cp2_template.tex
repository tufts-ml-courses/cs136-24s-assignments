\documentclass[10pt]{article}

%%% Doc layout
\usepackage{fullpage} 
\usepackage{booktabs}       % professional-quality tables
\usepackage{graphicx}       % figure graphics
\usepackage{microtype}      % microtypography
\usepackage{parskip}
\usepackage{times}


%% Hyperlinks always black, no weird boxes
\usepackage[hyphens]{url}
\usepackage[colorlinks=true,allcolors=black,pdfborder={0 0 0}]{hyperref}

%%% Math typesetting
\usepackage{amsmath,amssymb}

%%% Write out problem statements in purple, solutions in black
\usepackage{xcolor}
\newcommand{\officialdirections}[1]{{\color{purple} #1}}

%%% Avoid automatic section numbers (we'll provide our own)
\setcounter{secnumdepth}{0}


%% --------------
%% Header
%% --------------
\usepackage{fancyhdr}
\setlength{\headheight}{12.0pt}
\fancyhf{}
\fancyhead[C]{\ifnum\value{page}=1 CS 136 - 2024s - Coding Practical for Unit 2 (CP2) \else \fi}
\fancyfoot[C]{\thepage} % page number
\renewcommand\headrulewidth{0pt}
\pagestyle{fancy}


%% --------------
%% Begin Document
%% --------------
\begin{document}

~~\\ %% add vertical space

{\Large{\bf CP2 Report }}

\Large{\bf Collaboration Statement:}

TODO

Total hours spent: TODO

~~\\
~~\\
Links: 
\href{https://www.cs.tufts.edu/cs/136/2024s/cp2.html}{[CP2 instructions]} 
\href{https://www.cs.tufts.edu/cs/136/2024s/index.html#collaboration}{[Course collaboration policy]} 

\setcounter{tocdepth}{2}
\tableofcontents

\newpage

\subsection{1a}
\officialdirections{
Given a dataset of size $N$, how do we score the model's predictions? Translate the provided starter code for the score function of the MAP estimator into a mathematical expression involving our probabilistic model. Provide a function in terms of parameters $w_\text{MAP}, \beta$ and dataset ${x_n,t_n}_{n=1}^N$.}

1a: Solution
TODO YOUR SOLUTION HERE

\subsection{1b}
\officialdirections{Explain why different $\beta$ values might be preferred by different model orders when using MAP. (Hint: Execute run\_demo\_MAP.py at 3 different $\beta$ values: \{0.1, 1, 10\}. Look at the MAP estimator's scores on the training-set to see which $\beta$ is preferred by which order. Then study the provided visualization to get intuition about why.).}

1b: Solution
TODO YOUR SOLUTION HERE

\subsection{1c:}
\officialdirections{Compare the visuals produced by run\_demo\_PPE.py to those from the MAP estimator. What do you notice about the width of the 2-stddev intervals of the predictions at $x$ values far from the train data? Why does this suggest PPE might generalize better than MAP?}

1c Solution:
TODO YOUR SOLUTION HERE
\newpage 

\subsection{2a: }
\renewcommand{\figurename}{Fig.}
\renewcommand{\thefigure}{2a}
 \begin{figure}[!h]
     \centering
     \includegraphics[width=0.95\textwidth]{example-image-a.pdf}
     \label{fig:2a}
\caption{TODO YOUR CAPTION HERE. 
}%endcaption
 \end{figure}

\newpage 

\subsection{2b}
\renewcommand{\thefigure}{2b}
 \begin{figure}[!h]
     \centering
     \includegraphics[width=0.6\textwidth]{example-image-b.pdf}
     \label{fig:2b}
\caption{
TODO YOUR CAPTION HERE
}%endcaption
 \end{figure}

\newpage 

\subsection{3a}
\renewcommand{\thefigure}{3a}
 \begin{figure}[!h]
     \centering
     \includegraphics[width=0.95\textwidth]{example-image-c.pdf}
     \label{fig:3a}
     \caption{
TODO YOUR CAPTION HERE
}%endcaption
 \end{figure}

\newpage 

\subsection{3b}
\renewcommand{\thefigure}{3b}
 \begin{figure}[!h]
     \centering
     \includegraphics[width=0.6\textwidth]{example-image-c.pdf}
     \label{fig:3b}
\caption{
TODO YOUR CAPTION HERE
}%endcaption
 \end{figure}
 
\end{document}
