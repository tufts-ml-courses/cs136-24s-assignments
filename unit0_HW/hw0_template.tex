\documentclass[10pt]{article}

%%% Doc layout
\usepackage{fullpage} 
\usepackage{booktabs}       % professional-quality tables
\usepackage{microtype}      % microtypography
\usepackage{parskip}
\usepackage{times}

%% Hyperlinks always black, no weird boxes
\usepackage[hyphens]{url}
\usepackage[colorlinks=true,allcolors=black,pdfborder={0 0 0}]{hyperref}

%%% Math typesetting
\usepackage{amsmath,amssymb}

%%% Write out problem statements in purple, solutions in black
\usepackage{xcolor}
\newcommand{\officialdirections}[1]{{\color{purple} #1}}

%%% Avoid automatic section numbers (we'll provide our own)
\setcounter{secnumdepth}{0}

%% --------------
%% Header
%% --------------
\usepackage{fancyhdr}
\fancyhf{}
\fancyhead[C]{CS 136 - 2024s - HW0 Submission}

\fancyfoot[C]{\thepage} % page number
\renewcommand\headrulewidth{0pt}
\pagestyle{fancy}


%% --------------
%% Begin Document
%% --------------
\begin{document}

~~\\ %% add vertical space

{\Large{\bf Student Name: TODO}}

\Large{\bf Collaboration Statement:}

Total hours spent: TODO

I consulted the following resources:
\begin{itemize}
\item TODO
\item TODO
\item $\ldots$	
\end{itemize}

Links: 
\href{https://www.cs.tufts.edu/cs/136/2024s/hw0.html}{[HW0 instructions]} 
\href{https://www.cs.tufts.edu/cs/136/2024s/index.html#collaboration}{[collab. policy]} 

\tableofcontents

\newpage
\officialdirections{
\subsection*{Problem 1}
Imagine 3 sheets are shaken up together in a drawer. One sheet is drawn uniformly at random and put down on the ground. Only one side (the "face up" side) is visible.

Let $S$ be the random variable indicating which sheet is chosen. It is a discrete r.v. with 3 possible values:

\begin{itemize}
    \item BB : blue/blue sheet
    \item BO : blue/orange sheet
    \item OO : orange/orange sheet
\end{itemize}

Let $U$ be the random variable indicating which color is observed on the *face-up* side. This discrete r.v. has 2 possible values:

\begin{itemize}
    \item b : for blue
    \item o : for orange
\end{itemize}

}

\officialdirections{
\subsection*{1a: Problem Statement}
Compute the joint probability table for all possible configurations of your two random variables.

Write your answers as a 3x2 table (rows should indicate sheet values for $S$, columns should indicate face-up colors for $U$). Please write each probability entry as a simplified fraction.
}

\subsection{1a: Solution}

\begin{tabular}{r | l l}
         &  U = b  & U = o    \\ \hline 
S = BB  &  TODO  & TODO   \\
S = BO  &  TODO  & TODO    \\
S = OO  &  TODO &  TODO
\end{tabular}

\newpage
\officialdirections{
\subsection*{1b: Prompt}
You observe one random draw from this process: the chosen sheet's face-up side is blue.
What is the probability that the \textbf{face-down} side of the chosen sheet is orange?
}

\subsection{1b: Solution}
TODO YOUR SOLUTION HERE

\newpage
\officialdirections{
\subsection*{2a: Problem Statement}
Consider any possible joint distribution over $H, E_1, E_2$.

Suppose we only know the PMFs listed in each scenario (i) - (iii) above. Do we have enough information to calculate $p(H | E_1, E_2)$? Provide 1-2 sentences of justification.
}

\subsection{2a: Solution}
TODO YOUR SOLUTION HERE

\officialdirections{
\subsection*{2b: Problem Statement}
Now suppose $E_1$ and $E_2$ are \textbf{conditionally independent} given $H$.

For each set of PMFs above (i) - (iii), does this additional assumption allow us to calculate $p(H | e_1, e_2)$ ? Provide 1-2 sentences of justification. (If a case could already be handled affirmatively in 2a, just say ``already done successfully in 2a'').
}

\subsection{2b: Solution}
TODO YOUR SOLUTION HERE

\newpage
\officialdirections{
\subsection*{3a: Problem Statement}
Assume the scenario and model defined in the \href{https://www.cs.tufts.edu/cs/136/2024s/hw0.html#problem-3}{Problem 3 instructions}. 

What is the expected total score when using all $N$ dice? Write your answer as a function of $N$. Show your work.
}

\subsection{3a: Solution}

TODO YOUR SOLUTION HERE

\end{document}
