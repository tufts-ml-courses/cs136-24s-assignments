\documentclass[12pt]{article}

%%% Doc layout
\usepackage{fullpage} 
\usepackage{booktabs}       % professional-quality tables
\usepackage{graphicx}       % figure graphics
\usepackage{microtype}      % microtypography
\usepackage{parskip}
\usepackage{times}

\usepackage{caption}
\captionsetup{font=large}

%% Hyperlinks always black, no weird boxes
\usepackage[hyphens]{url}
\usepackage[colorlinks=true,allcolors=black,pdfborder={0 0 0}]{hyperref}

%%% Math typesetting
\usepackage{amsmath,amssymb}

%%% Write out problem statements in purple, solutions in black
\usepackage{xcolor}
\newcommand{\officialdirections}[1]{{\color{purple} #1}}

%%% Avoid automatic section numbers (we'll provide our own)
\setcounter{secnumdepth}{0}

%% --------------
%% Header
%% --------------
\usepackage{fancyhdr}
\setlength{\headheight}{12.0pt}
\fancyhf{}
\fancyhead[C]{\ifnum\value{page}=1 CS 136 - 2024s - Coding Practical for Unit 1 (CP1) \else \fi}
\fancyfoot[C]{\thepage} % page number
\renewcommand\headrulewidth{0pt}
\pagestyle{fancy}


%% --------------
%% Begin Document
%% --------------
\begin{document}

~~\\ %% add vertical space

{\Large{\bf CP1 TEMPLATE: [TODO YOUR NAME HERE] }}

\Large{\bf Collaboration Statement:} TODO


Total hours spent: TODO

~~\\
~~\\
Links: 
\href{https://www.cs.tufts.edu/cs/136/2024s/cp1.html}{[CP1 instructions]} 
\href{https://www.cs.tufts.edu/cs/136/2024s/index.html#collaboration}{[Course collaboration policy]} 

\setcounter{tocdepth}{2}
\tableofcontents

\newpage

\officialdirections{
\subsection*{1a: Figure for Experiment 1: Test-set Log Likelihood vs Train Set Size }
}

\subsection{1a: Solution}

\begin{figure}[!h]
     \centering
     \includegraphics[height=9cm]{example-image-a.jpg} % TODO replace with your figure file
     \caption{ONE SENTENCE CAPTION HERE}
     \label{fig:fig1a}
\end{figure}


\newpage 
\officialdirections{
\subsection*{1b: Problem Statement}

\emph{\large
 Will the ML estimator eventually be clearly superior to the others, given enough training data? Why or why not? (Hint: what happens as $N \rightarrow \infty$).
}}

\subsection{1b: Solution}

TODO YOUR ANSWER HERE

\officialdirections{
\subsection*{1c: Problem Statement}
\emph{\large
What test-set log likelihood performance would you get if you simply predicted the next word using a uniform probability distribution over the vocabulary? Please report a concrete numerical value, comparable to the scores in Fig 1a above. Does the ML estimator always beat this "dumb" baseline?
}}

\subsection{1c: Solution}

TODO YOUR ANSWER HERE

\newpage 
\officialdirections{
\subsection*{2a: Figure for Experiment 2: Selecting a Value of Hyperparameter $\alpha$}
}

\subsection{2a: Solution}

\begin{figure}[!h]
     \centering
     \includegraphics[height=9cm]{example-image-b.jpg} % TODO replace with your figure file
     \caption{ONE SENTENCE CAPTION HERE}
     \label{fig:fig1a}
\end{figure}

\officialdirections{
\subsection*{2b: Takeaway from Figure 2a}
\emph{\large 
Across all 3 dataset sizes, compare the best $\alpha$ value as selected by the evidence to the $\alpha$ selected by looking directly at test-set likelihood. Do these peaks occur at ``similar'' locations, if we define similar to x as within (0.2x, 5x)? What does this suggest about the viability of using the evidence to select $\alpha$ that might predict well on future state-of-the-union speeches?
}}

\subsection{2b: Solution}

YOUR ANSWER HERE

\end{document}
