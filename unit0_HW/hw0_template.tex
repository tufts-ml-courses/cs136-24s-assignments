\documentclass[10pt]{article}

%%% Doc layout
\usepackage{fullpage} 
\usepackage{booktabs}       % professional-quality tables
\usepackage{microtype}      % microtypography
\usepackage{parskip}
\usepackage{times}

%% Hyperlinks always black, no weird boxes
\usepackage[hyphens]{url}
\usepackage[colorlinks=true,allcolors=black,pdfborder={0 0 0}]{hyperref}

%%% Math typesetting
\usepackage{amsmath,amssymb}

%%% Write out problem statements in purple, solutions in black
\usepackage{xcolor}
\newcommand{\officialdirections}[1]{{\color{purple} #1}}

%%% Avoid automatic section numbers (we'll provide our own)
\setcounter{secnumdepth}{0}

%% --------------
%% Header
%% --------------
\usepackage{fancyhdr}
\fancyhf{}
\fancyhead[C]{CS 136 - 2024s - HW0 Submission}

\fancyfoot[C]{\thepage} % page number
\renewcommand\headrulewidth{0pt}
\pagestyle{fancy}


%% --------------
%% Begin Document
%% --------------
\begin{document}

~~\\ %% add vertical space

{\Large{\bf Student Name: TODO}}

\Large{\bf Collaboration Statement:}

Total hours spent: TODO

I consulted the following resources:
\begin{itemize}
\item TODO
\item TODO
\item $\ldots$	
\end{itemize}

Links: 
\href{https://www.cs.tufts.edu/cs/136/2024s/hw0.html}{[HW0 instructions]} 
\href{https://www.cs.tufts.edu/cs/136/2024s/index.html#collaboration}{[collab. policy]} 

\tableofcontents

\newpage
\officialdirections{
\subsection*{Problem 1}
A desk drawer contains 3 sheets of paper, identical in all respects (size etc) except:

\begin{itemize}
\item sheet 1: both sides are colored orange
\item sheet 2: both sides are colored blue
\item sheet 3: one blue side and one orange side
\end{itemize}

The sheets are shaken up together in the drawer. One sheet is drawn uniformly at random and put face down on the ground. Only one side (the "face up" side) is visible.

You observe one random draw from this process: the chosen sheet's face-up side shows a blue color.

Naturally, you are still uncertain about the color of the \textbf{face-down} side of the chosen sheet.
}

\subsubsection*{Probabilistic model}
Let $\boldsymbol{S}$
 be the random variable indicating which sheet is chosen. It is a discrete r.v. with 3 possible values:

\begin{itemize}
    \item BB : blue/blue sheet
    \item BO : blue/orange sheet
    \item OO : orange/orange sheet
\end{itemize}


Let $\boldsymbol{U}$
 be the random variable indicating which color is observed on the face-up side. This discrete r.v. has 2 possible values:
\begin{itemize}
    \item b : for blue
    \item o : for orange
\end{itemize}


\officialdirections{
\subsection*{1a: Problem Statement}
Compute the joint probability table for all possible configurations of your two random variables.

Write your answers as a 3x2 table (rows should match to sheets, columns should match to face-up colors).
}

\subsection{1a: Solution}

TODO YOUR SOLUTION HERE
\begin{center}
    \begin{tabular}{|c|c|c|}
        \hline
        \textbf{Row/Column} & \textbf{Column 1} & \textbf{Column 2} \\
        \hline
        \textbf{Row 1} & Data 1,1 & Data 1,2 \\
        \hline
        \textbf{Row 2} & Data 2,1 & Data 2,2 \\
        \hline
        \textbf{Row 3} & Data 3,1 & Data 3,2 \\
        \hline
    \end{tabular}
\end{center}

\officialdirections{
\subsection*{1c: Prompt}
You observe one random draw from this process: the chosen sheet's face-up side is blue.

Naturally, you are still uncertain about the color of the face-down side of the chosen sheet.

What is the probability that the face-down side of the chosen sheet is orange?
}
\subsection{1c: Solution}

TODO YOUR SOLUTION HERE

\officialdirections{
\subsection*{Problem 2}

Consider a joint model of 3 random variables:

\begin{itemize}
\item $H$ : a discrete r.v. with K possible values
\item $E_1$: a discrete r.v.  with A possible values
\item $E_2$: a discrete r.v. with B possible values
\end{itemize}

Imagine three possible sets of PMFs we could know completely:

\begin{itemize}
\item (i) $p(H)$, $p(E_1, E_2)$, $p(E_1|H), p(E_2|H)$
\item (ii) $p(H)$, $p(E_1, E_2)$, $p(E_1, E_2|H)$
\item (iii) $p(H)$, $p(E_1|H), p(E_2|H)$
\end{itemize}
}

\officialdirections{
\subsection*{2a: Problem Statement}
Consider any joint distribution over $H, E_1, E_2$.

Suppose we only know the PMFs listed in each scenario (i) - (iii) above. Do we have enough information to calculate $p(H | E_1, E_2)$ ? Provide 1-2 sentences of justification.
}

\subsection{2a: Solution}
TODO YOUR SOLUTION HERE

\officialdirections{
\subsection*{2b: Problem Statement}
Now suppose $E_1$ and $E_2$ are \textbf{conditionally independent} given $H$.

For each set of PMFs above (i) - (iii), does this additional assumption allow us to calculate $p(H | E_1, E_2)$ ? Provide 1-2 sentences of justification.
}

\subsection{2b: Solution}
TODO YOUR SOLUTION HERE

\officialdirections{
\subsection*{Problem 3}

Consider rolling $N$
 dice. Each is a standard 6-sided die, with each side showing a distinct number in the set $\{1,2,3,4,5,6\}$. Each die can be modeled as independent and identically distributed with uniform chance for each of the sides.
You play a game where you get the following points for the outcome of each die:
\begin{itemize}
\item 1000 if you see a 1
\item 500 if you see a 5
\item 0 otherwise
\end{itemize}
Your total score is the sum of the points from each individual die.
}


\officialdirections{
\subsection*{3a: Problem Statement}
What is the expected total score when using all $N$
 dice? Write your answer as a function of $N$
. Show your work.
}

\subsection{3a: Solution}
TODO YOUR SOLUTION HERE


\end{document}
